\section{Tools}
\label{sec:tools}

In the following, you will find short descriptions of some regularly used tools that are supplied with \CM.
Each of them can support you in either setting up your simulations or dealing with geometry related tasks.
Additionally, it is recommended to take a look into the \href{https://github.com/KIT-IBT}{IBT GitHub} for other useful tools, \eg rule based fiber generation algorithms (\href{https://github.com/KIT-IBT/RESILIENT}{RESILIENT}, \href{https://github.com/KIT-IBT/LDRB_Fibers}{LDRB Fibers}), ventricular coordinate system (\href{https://github.com/KIT-IBT/Cobiveco}{Cobiveco}), and the powerful \href{https://github.com/KIT-IBT/vtkToolbox}{vtkToolbox} for MATLAB.

\subsection{BidomainMatrixGenerator}
\label{tools:BidomainMatrixGenerator}

This tool generates matrices and vectors needed for mono- and bidomain FEM simulations with acCELLerate. 
That includes stiffness and mass matrices, materials vector, and index set. 
Output files are in PETSc format.
When using the \verb|-gzip| option, all output file will have the additional suffix .gz. You can use these files with acCELLerate without decompressing them!
The only mandatory argument is the geometry as a \nameref{subsubsec:VTK}.
It should contain the data arrays \verb|Material| and one of the following arrays containing information about the fiber orientation:
\begin{itemize}
    \item \verb|Phi|, \verb|Theta| are a specific angle measurement to define fiber orientation with a binary quantization of the angle using 8 bits or 1 Byte such that $\phi = \Tilde{\phi} \cdot \frac{254}{\pi}$ and $\theta = \Tilde{\theta} \cdot \frac{254}{\pi}$, where $\Tilde{\phi}$ and $\Tilde{\theta}$ are spherical coordinates given in radians.
    An orientation vector is calculated using
    \begin{equation}
        \vec f_0 = \mqty( -\cos \Tilde{\theta} \cos\Tilde{\phi} \\ -\cos\Tilde{\theta} \sin\Tilde{\phi} \\ \sin\Tilde{\theta} )
    \end{equation}
    $\phi = \theta = 255$ represents the isotropic case.
    Aligning the fibers with the x-axis can be done with $\phi = 0$ and $\theta = 127$.

    \item If you already have an orientation vector, you can supply it with an array called either \verb|Fiber| or \verb|Orientation|.
\end{itemize}

\begin{lstlisting}[language=bash,caption=Runtime options for BidomainMatrixGenerator]
BidomainMatrixGenerator <.vt?> [-o <prefix>] [-v]
    [-bath <mask> [mat.def]]
    [-mono <mat.def> [mat2.def]]
    [-bi <intra.def> <extra.def>]
    [-tissue <list>] [-freq <freq>]
    [-combined] [-coupled <betaCm_dt>] [-gauss <n>]
    [-fullLumping] [-massLumping <theta>]
    [-iso] [-gzip] [-ignoreMissing]

-v --- Enable additional informational (verbose) output.

-bath <mask> [mat.def] --- All elements matching <mask> will only be part of the extracellular domain. Requires [-bi ...]. If [mat.def] is not given, <extra.def> is used.

-mono <mat.def> [mat2.def] --- Creates monodomain matrix. If [mat2.def] is given, the conductivity will be the harmonic mean of the two given conductivities for each cell (parallel circuit).

-bi <intra.def> <extra.def> --- Creates bidomain matrices with the intra- and extracellular conductivities from the given material files.

-combined --- Creates bidomain matrices where intracellular matrix is already added to extracellular matrix.

-coupled <betaCm_dt> [theta] --- Create fully coupled bidomain matrices (LHS and RHS) <betaCm_dt> is (beta*C_m)/dt. theta is the time discretization weight (0=explicit, 1=implicit, 0.5=Crank-Nicolson, default = 0).

-gauss <n> --- Create Gauss interpolation and integration matrices. <n> is the number of Gauss points. Default integration rules are provided.

-fullLumping --- Use full mass lumping and multiply intra/mono matrix with inverse of lumped mass matrix. Will not save mass matrix separately. Does not apply with -coupled. Use -massLumping 0 instead.

-massLumping <theta> --- Use mass lumping. Saved mass matrix will be weighted sum: M* = (theta)M + (1-theta)*M_lumped Usually, you would choose theta in [0,1]. Different choices may work, however.

-tissue <list> --- Comma-separated list of tissue classes in order of their, precedence when creating the tissue vector. Highest precedence first. Default: smaller numbers have higher precedence.

-freq <freq> --- Evaluate conductivities at frequency <freq> if supported.

-iso --- Ignore fiber orientation (isotropic case).

-ignoreMissing --- Ignore missing intra conductivities, i.e., set them to 0. Only recommended to use with -bi.

-bds <value> --- CV stabilization factor beta*Delta_s.

-gzip --- Compress output files.

-o <prefix> --- Prefix for saving the resulting files. Defaults to the name of the input geometry sans extension. Depending on the given options, the following files may be created:
           <prefix>.vec - Material class vector.
           <prefix>.mat - Monodomain matrix.
           <prefix>.mass.mat - Mass matrix.
           <prefix>.intra.mat - Bidomain intra matrix.
           <prefix>.extra.mat - Bidomain extra matrix.
           <prefix>.i+e.mat - Bidomain combined matrix.
           <prefix>.n2e.mat - Node-to-element Gauss point (interpolation) matrix.
           <prefix>.e2n.mat - Element-to-node Gauss point (integration) matrix.
           <prefix>.lhs.mat - Coupled LHS matrix.
           <prefix>.rhs.mat - Coupled RHS matrix.
           <prefix>.is - Index set of intracellular domain.
\end{lstlisting}

\subsection{PETScVec2VTK}
\label{tools:PETScVec2VTK}

Converts a PETSc vector as e.g. provided by acCELLerate to a VTK unstructured grid (.vtu) or VTK poly data (.vtp) file for postprocessing or visualization. 
The input VTK file should be the same that was used to generate the matrix using \nameref{tools:BidomainMatrixGenerator}.
The python script \nameref{tools:ConvertPETScDir2VTK} can be used to convert an entire directory at once.
\begin{lstlisting}[language=bash,caption=Syntax for PETScVec2VTK]
PETScVec2VTK <vector filename> <VTK filename (.vtp/.vtu) in>  <VTK filename (.vtp/.vtu) out>
	[--arrayName]	Name of the VTK array to be appended (default: Results)
	[--delete]	Delte PETScVec after conversion
\end{lstlisting}

\subsection{ConvertT4toT10}
\label{tools:ConvertT4toT10}

Use to convert linear elements to quadratic elements based on \nameref{nodeFile} and \nameref{eleFile}.

\begin{lstlisting}[language=bash,caption=Syntax for ConvertT4toT10]
ConvertT4toT10 <Nodes_File.node> <Elements_File.ele> <Output>
\end{lstlisting}

\subsection{VTPExtractSurfaceFromMesh}
\label{tools:VTPExtractSurfaceFromMesh}

Given a .node and .ele file of a geometry as well as a surface in .vtp/.stl format, you can use this tool to generate an appropriate .sur file with matching node indexing for use in \CM.
The \verb|-append| option can be used to fill a single .sur file with multiple surfaces in case of repeated execution of \verb|VTPExtractSurfaceFromMesh|.

\begin{lstlisting}[language=bash,caption=Syntax for VTPExtractSurfaceFromMesh]
VTPExtractSurfaceFromMesh <nodes_file.node> <elements_file.ele> <vtp.file> <element_index> <surface_index> <output_prefix> 
Options:
	-free_surface_element_index <val>
	-unit <val>
	-append
	-first_index <val>
	-export_nodes_indices <exp_nodes_file.node>
	-export_stand_alone <standalone_prefix>

\end{lstlisting}

\subsection{FixNodes}
\label{tools:FixNodes}

Can be used to apply Dirichlet boundary conditions to nodes specified in fixation.list.

\begin{lstlisting}[language=bash,caption=Syntax for FixNodes]
FixNodes <input.node> <output.node> <fixation.list> 
	-val <val> (optional) 
	-keep_original (optional) Modify only nodes from list, if not set, all 
    other nodes are set to 0 (unfixed)
\end{lstlisting}


\section{Scripts}
\label{tools:Python}

\subsection{VTK2tetgen}
\label{tools:VTK2tetgen}

Convert a vtk file to .node and .ele files for CardioMechanics.
If fiber orientation is given with the appropriate \verb|Fiber|, \verb|Sheet|, and \verb|Sheetnormal| data arrays, a .bases file is generated as well.

\begin{lstlisting}[language=bash,caption=Syntax for VTK2tetgen.py]
usage: VTK2tetgen.py [-h] [-outfile OUTFILE] [-scale SCALE]
                     [-fixMaterial FIXMATERIAL [FIXMATERIAL ...]]
                     mesh

positional arguments:
  mesh                  VTK input file.

optional arguments:
  -h, --help            show this help message and exit
  -outfile OUTFILE      Set an alternative output filename PREFIX if desired.
  -scale SCALE          Scale the VTK input file by a factor SCALE.
  -fixMaterial FIXMATERIAL [FIXMATERIAL ...]
                        Fix nodes if they are connected to cells that match 
                        the given mask in "Material". Example: "-
                        fixMaterial 32 162" 
\end{lstlisting}


\subsection{ConvertPETScDir2VTK}
\label{tools:ConvertPETScDir2VTK}

ConvertPETScDir2VTK.py automatically converts all *.vec files in the provided directory to *.vtu format and creates a *.pvd file containing all results ready to be viewed in ParaView.

\begin{lstlisting}[language=bash,caption=Syntax for ConvertPETScDir2VTK.py]
Usage: ConvertPETScDir2VTK.py [options]

Options:
  --version             show programs version number and exit
  -h, --help            show this help message and exit
  -d PATH, --directory=PATH
                        Directory containing simulation results
  -g GEOFILE, --geometry=GEOFILE
                        Geometry file (*.vtu) used for the simulation
  -o FILENAME, --outfile=FILENAME
                        Filename of the resulting *.pvd file
  -c, --clean           Choose wether or not to delete converted *.vec files
\end{lstlisting}


\subsection{tuneCV}
\label{tools:tuneCV}

This script can be used to tune conductivity values to achieve a desired conduction velocity on tetrahedral meshes based on the method described in \cite{costa2013automatic}.
It requires acCELLerate, \href{https://gmsh.info}{Gmsh}, and VTK to be installed.
If your Gmsh root directory differs from the default path used on MacOS, please provide it using one of the parameters of the script.
Conduction velocity is only evaluated in one direction using a planar wavefront.
For anisotropic cases, you have to do multiple runs with the desired conduction velocity.
Keep in mind that this evaluation is done on a very simple geometry, meaning the actual CV in your heart geometries may differ depending on a plethora of circumstances.
Nevertheless, this tool should still leave you with a good estimate.

\begin{lstlisting}[language=bash,caption=Syntax for tuneCV.py]
usage: tuneCV.py [-h] [-resolution RESOLUTION] [-length LENGTH] 
                 [-velocity VELOCITY]
                 [-gi GI] [-ge GE] [-gm GM] [-beta BETA] [-cm CM] [-
                 sourceModel {mono}]
                 [-tol TOL] [-converge] [-maxit MAXIT] [-dt DT] [-theta 
                 {0,0.5,1}]
                 [-stol STOL] [-lumping {True,False}]
                 [-activationThreshold ACTIVATIONTHRESHOLD] [-log LOG]
                 [-gmshexec GMSHEXEC]
                 {BeelerReuter,TenTusscher2,CourtemancheEtAl,OHaraRudy,HimenoEtAl,KoivumaekiEtAl,GrandiEtAlVentricle,GrandiEtAlAtrium}
                 amplitude duration

positional arguments:
  {BeelerReuter,TenTusscher2,CourtemancheEtAl,OHaraRudy,HimenoEtAl,KoivumaekiEtAl,GrandiEtAlVentricle,GrandiEtAlAtrium}
                        Ionic model
  amplitude             Stimulus amplitude
  duration              Stimulus duration in s

optional arguments:
  -h, --help            show this help message and exit
  -resolution RESOLUTION mesh resolution in mm (default is 0.5 mm)
  -length LENGTH        length of strand in mm (default is 20.0 mm)
  -velocity VELOCITY    desired conduction velocity in m/s 
                        (default: 0.67 m/s)
  -gi GI                initial value intracellular conductivity in S/m 
                        (default: 0.174 S/m)
  -ge GE                initial value extracellular conductivity in S/m 
                        (default: 0.625 S/m)
  -gm GM                initial value monodomain conductivity (harmonic 
                        mean of gi/ge) in S/m (default: 0.136 S/m)
  -beta BETA            Membrane surface-to-volume ratio (default: 140000 
                        m^-1)
  -cm CM                Membrane capacitance per unit area 
                        (default: 0.01 Fm^-2)
  -sourceModel {mono}   pick type of electrical source model 
                        (default is monodomain)
  -tol TOL              Percentage of error in conduction velocity 
                        (default: 0.01)
  -converge             Iterate until converged velocity (default: False)
  -maxit MAXIT          Maximum number of conduction velocity iterations 
                        (default: 20)
  -dt DT                Integration time step on seconds (default: 1e-5 s)
  -theta {0,0.5,1}      Choose time stepping method 0: explicit, 0.5: CN, 
                        1: implicit (default: 0.5)
  -stol STOL            Solver tolerance (default: 10^-6)
  -lumping {True,False} Use mass lumping (default: False)
  -activationThreshold ACTIVATIONTHRESHOLD
                        Threshold to record LAT (default: -0.02 V)
  -log LOG              Generate/append to a log file.
  -gmshexec GMSHEXEC    Use this option to define an alternative Gmsh path.
\end{lstlisting}

\subsection{CircWholeHeart}
\label{tools:CircWholeHeart}

This script is an implementation of the basic 4-chamber circulatory system model described in Section \ref{plugin:Circulation}.
The script has not much use by itself, but is ready to be used with the Python library \href{https://github.com/FrancescoRegazzoni/cardioemulator}{cardioemulator} by Francesco Regazzoni to construct zero-dimensional emulators of the cardiac electromechanical function.

\subsection{createSimDir}
\label{tools:createSimDir}

This script helps you to set up a directory for electromechanical simulations with \CM.
All arguments are optional.
However, if you already know which material numbers, conductivities, and ionic models you want to use, you can provide lists in the associated arguments.
Other parameters for the monodomain equation such as membrane surface-to-volume ratio, membrane capacitance and integration time step can be given as well.
This will automatically create the correct entries in the respective files.
If no arguments are given, the files will be populated with some default arguments.
Files that require the geometry need to be set up by the user.
The location of files that are not generated by the script are indicated by the command line output.

\begin{lstlisting}[language=bash,caption=Syntax for createSimDir.py]
usage: createSimDir [-h] [-material [MATERIAL ...]]
                    [-model [{BeelerReuter,TenTusscher2,CourtemancheEtAl,OHaraRudy,HimenoEtAl,KoivumaekiEtAl,GrandiEtAlVentricle,GrandiEtAlAtrium} ...]]
                    [-sigma [SIGMA ...]] [-beta BETA] [-cm CM] [-time TIME]
                    [-dt DT] [-theta {0,0.5,1}] [-sourceModel {mono}]
                    [-activationThreshold ACTIVATIONTHRESHOLD] [-dir DIR]

options:
  -h, --help            show this help message and exit
  -material [MATERIAL ...]
                        List of material numbers (default: 0)
  -model [{BeelerReuter,TenTusscher2,CourtemancheEtAl,OHaraRudy,HimenoEtAl,KoivumaekiEtAl,GrandiEtAlVentricle,GrandiEtAlAtrium} ...]
                        List of ionic models (default: ElphyDummy)
  -sigma [SIGMA ...]    List of monodomain conductivities in S/m (default:
                        0.136 S/m)
  -beta BETA            Membrane surface-to-volume ratio (default: 140000
                        m^-1)
  -cm CM                Membrane capacitance per unit area (default: 0.01
                        Fm^-2)
  -time TIME            Simulation time in s (default: 0.8 s)
  -dt DT                Integration time step in seconds (default: 1e-5 s)
  -theta {0,0.5,1}      Choose time stepping method 0: explicit, 0.5: CN, 1:
                        implicit (default: 0.5)
  -sourceModel {mono}   pick type of electrical source model (default is
                        monodomain)
  -activationThreshold ACTIVATIONTHRESHOLD
                        Threshold to record LAT (default: -0.02 V)
  -dir DIR              Name of top level directory (default: SimDir)
\end{lstlisting}

\subsection{ClipVTUwithVTU}
\label{tools:ClipVTUwithVTU}

This script is a short post-processing tool for vtu files.
Using a clipped version of the mesh created in \eg ParaView, you can clip an entire time series while preserving all elements even when they move out of the clipped plane. 

\begin{lstlisting}[language=bash,caption=Syntax for ClipVTUwithVTU.py]
usage: ClipVTUwithVTU.py [-h] [-outfile OUTFILE] [-start START] [-stop STOP] [-step STEP] mesh clip

positional arguments:
  mesh              vtk/vtu/vtp data you want to clip.
  clip              clipped version of <mesh> created in e.g. ParaView invtu/vtk/vtp.

optional arguments:
  -h, --help        show this help message and exit
  -outfile OUTFILE  prefix of the results file.
  -start START      start value for the time series.
  -stop STOP        stop value for the time series.
  -step STEP        increment of the time series.

\end{lstlisting}